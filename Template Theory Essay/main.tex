% Turabian Formatting for Research Papers Template, 2018/08/06
%
% Developed using the turabian-formatting package (2018/08/01), available through CTAN: 
% -------------------------------------------
% http://www.ctan.org/pkg/turabian-formatting
% -------------------------------------------
%
% Additional document class formatting options:
%
% raggedright: ragged right formatting without hyphenations
% authordate: support for the author-date citation style
% endnotes: support for endnotes


% If you want the raggedright or endnotes formatting options, you can stipulate that in brackets here. Each additional formatting option is delimited by commas.
\documentclass[authordate]{turabian-researchpaper}


\usepackage[utf8]{inputenc}
\usepackage{csquotes, ellipsis}

% Specify paper size with geometry package
\usepackage[pass, letterpaper]{geometry}

% For citations, use the biblatex-chicago package
\usepackage{biblatex-chicago}
\addbibresource{works-cited.bib}

% Information for title page
\title{David Hume}
\subtitle{The Conditional Monarchist}
\author{Kevin Navarrete-Parra}
\course{Political Science 782R}
\date{\today}


\begin{document}

% This line generates the title page from the information above. 
\maketitle

% \tableofcontents generates a toc 

% \section, \subsection, etc useful for essay sections

% use \begin{quotation} and \end{quotation} for block quotes

\autocite[93]{Hume_1985}

David Hume’s political philosophy proves particularly challenging to nail down because he persistently qualifies and hedges his statements, often making it challenging to identify the hard stances he takes in his writings. In his Essays: Moral, Political, and Literary, Hume defies the typical pattern many political theorists follow by not explicitly defining a regime typology—though he does focus heavily on distinguishing republics from monarchies throughout the work. Although the distinction between those two regime types may not serve the same purpose as regime typologies in other political theory writings, Hume’s discussion points to their salience regarding political societies. Additionally, his protracted treatment of republics and monarchies begs the question of preference, which I believe falls in favor of monarchies, though to a limited extent and only under certain power constraints. 

Hume’s preference for limited monarchy partially shows its head in the Essay “Whether the British Government Inclines More to Absolute Monarchy, or to a Republic,” where he engages his readers to evaluate arguments regarding Britain’s political system. Before examining what he concluded, it is crucial to frame the Essay like Hume did: pointing out the imprudence associated with prophesying remote events or consequences \autocite[47]{Hume_1985}. This principle calls into question the degree to which one should take the Essay’s conclusions seriously—at least as far as those conclusions relate to predictions about the far-off future. 

Although Hume warns readers against prophesying, he spends the Essay’s bulk discussing Britain’s regime trend at the time of writing, arguing that the crown is gradually increasing in power \autocite[51]{Hume_1985}. Granted, the country’s sentiments ran rapidly towards popular government, according to Hume \autocite[51]{Hume_1985}, but the increasing tax revenue made the crown’s growth outpace the popular government. As we see below, monarchical governments’ increasing tax revenue means their present inferiority to republics will eventually be null. 

Hume ends the Essay by discussing how he would like to see his government die. He states, “I would frankly declare, that, though liberty be preferable to slavery, in almost every case; yet I should rather wish to see an absolute monarch than a republic in this land.”\autocite[52]{Hume_1985} This quote is important because Hume shows his preference for free government, yet he recognizes absolute monarchy’s fundamental stability during difficult times, such as during a government’s decadence. One must not conflate what Hume says here with the distinction between monarchies and republics because slavery and liberty are not these regimes’ corollaries. Indeed, Hume points out much earlier that liberty is not necessarily associated with any government, whether republican or monarchical \autocite[9-10]{Hume_1985}, so his preference for liberty later does not betray an unqualified desire for republicanism. Hume’s closing statement in this Essay points to that fact because he warns against both government’s dangers.\autocite[53]{Hume_1985}

Shifting now to Hume’s arguments in the Essay “Of Civil Liberty,” one finds him oscillating between monarchies and republics once again. However, he points out that the world is not yet old enough to draw general conclusions about political matters, highlighting flaws in Machiavelli’s \emph{The Prince} \autocite[87-9]{Hume_1985}. This theme pervades throughout the Essay, pointing readers to flaws in the Ancients’ views on liberty’s relation to arts and sciences and errors other thinkers made regarding liberty’s impact on commerce \autocite[89-93]{Hume_1985}. Although Hume’s intention in this essay was to “make a full comparison of civil liberty and absolute government, and to show the great advantages of the former above to the latter” \autocite[89]{Hume_1985}, he quickly shifts course to discussing the merits of different regime types. 

Importantly, Hume discusses modern government’s developments over time, pointing out that monarchies developed into governments of laws, even if popular governments exceed the former in gentleness and stability \autocite[94]{Hume_1985}. However, one should not take this statement as a clear indication of Hume’s preference for republics over monarchies. Instead, this statement indicates that both regime types exist on a spectrum of perfectibility, and monarchies have grown dramatically over time.

Additionally, as Hume’s closing statements in this Essay indicate, monarchical governments have an element of improvement, whereas popular governments have a source of degeneracy.\autocite[95-6]{Hume_1985} Monarchies, Hume argues, will improve to such a degree that time will bring them closer to equality with popular government as long as they solve their issues with predatory financiers and their “expensive, unequal, arbitrary, and intricate method of levying [taxes]” \autocite[95]{Hume_1985} That is, when an enlightened prince or minister arises who dispenses with the ancient custom of beggaring their citizens, monarchies’ upward trend will lead them to equalization in relation to republics. 

On the other hand, republics betray a source of degeneracy by contracting public debts and mortgaging public revenues.\autocite[95]{Hume_1985} More specifically, this problem “consists in the practice of contracting debt, and mortgaging the public revenues, by which taxes may, in time, become altogether intolerable, and all the property of the state be brought into the hands of the public.” \autocite[96]{Hume_1985} This is a modern development, according to Hume, and it was nowhere to be found in ancient republics (1987, 95). More importantly, this problem threatened all free governments, including Britain’s liberty at the time, according to Hume.\autocite[96]{Hume_1985}

If Hume is correct when he contrasts public finance and revenue in monarchies and republics, he likely favors the former over the latter. Of course, he is not a staunch monarchist in any way. Instead, Hume carefully considers his political beliefs, weighing every argument before deciding. In this context, Hume first identifies the crown’s increasing tax base, arguing that this pivotal fact makes Britain more monarchical than republican.\autocite[51]{Hume_1985} Moreover, he points to tax collection as monarchies’ principal element of improvement, making, in this context, Britain’s upward trend all more notable. Therefore, if monarchies are trending toward improvement and republics are degenerating, Hume’s preference lies with the former, assuming the rule of law limits these monarchies. Moreover, as we established above, the assumption is sound given Hume’s argument stating that “monarchical government seems to have made the greatest advances towards perfection. It may now be affirmed that civilized monarchies, what was formerly said in praise of republics alone, that they are a government of Laws, not of Men.”\autocite[94]{Hume_1985} 

However, despite Hume’s implicit preference for limited monarchies, he is not a dogmatic monarchist. Instead, Hume is simply aligning himself with the regime type he believes to be more stable at the time. Presented with alternate information, it would not be difficult to conceive of him favoring republics over monarchies, for in the end, his stronger preference lies with governments founded on liberty over those founded on slavery, arguing that “liberty is the perfection of civil society .\autocite[47]{Hume_1985} Of course, there are exceptions to this statement. However, any government that best suits liberty’s development is preferable, and the trend at the time indicated that limited monarchies are most hospitable to perfectibility. 

As we can see from these two Essays, Hume prefers monarchies to republics, given that the rule of law limits the monarchy. However, considering Hume’s stance against prophesying, one should not take this view as too rigid, for Hume would just as likely shift under different circumstances. Ultimately, Hume supports whatever regime is most capable of perfectibility through liberty, which, at the time, happens to be monarchical governments limited by the rule of law. 

\clearpage
\printbibliography

\end{document}
